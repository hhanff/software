%===============================================
%================KAPITEL 2======================
%================================================

\setcounter{secnumdepth}{2}	%	ab hier f�ngt die Nummerierung der Kapitel an

\chapter{Philips Semiconductors Sophia Antipolis}


% -- Seitennummerierung auf Arabische Zahlen zur�cksetzen (1,2,3) 
\pagenumbering{arabic}
%\initfloatingfigs


\section{Sophia Antipolis}
\paragraph{Der Technologiepark}
Gegr�ndet wurde der Technologiepark Sophia Antipolis 1969 von einem non-profit Unternehmen das �ber all die Jahre auch die Vermarktung des Parks �bernommen hat. Mittlerweile umfasst der Park 2300 Hektar in dem sich 1260 Unternehmen angesiedelt haben. 25922 direkte Arbeitspl�tze wurden inzwischen geschaffen. 

\paragraph{Forschung}
Die Universit�t Nizza - Sophia Antipolis und das CNRS sowie die zahlreichen Forschungslabore und technischen Institute bilden einen Forschungspool in Industrien�he.

\paragraph{Medizinische Forschung, Chemie und Biotechnologie}
Die Biotechnologie, Gesundheits- und Agrochemie umfassen rund 60 Unternehmen. Nur einige der f�hrenden Unternehmen sind dabei Rh�ne-Poulenc Agro, Dow Agrosience, Smith Kline Beecham Clinical Laboratories, Rohm and Corning, NMT Neuroscience Implants, S.A. und Allergan Europe.

\paragraph{Geowissenschaften}
Die Gebiete Neue Energien, Umwelt und Geowissenschaten besch�ftigen 250 Menschen, die in �ffentlichen und privaten, kleinen bis mittelgro�en Firmen arbeiten. In den ersten beiden genannten Gebieten findet man Firmen wie Geolab oder IMRA Europe.


\paragraph{Computerwissenschaften, Elektronik, Netzwerk und Kommunikation}
Computerwissenschaften, Elektronik, Netzwerk und Kommunikation stellen etwa 25\% der Unternehmen und Arbeitspl�tze. Unter Ihnen so bekannte Firmen wie Air France, AT\&T France, Philips Semiconductors, Bosch T�l�com, Compaq, Thomson, Siemens, IBM, Texas Instruments, W3C, Infineon, Schneider Electronics und Hewlett and Packard.


\section{Philips Semiconductors - Geschichte des Unternehmens}

\begin{figure}[H]
	\begin{center}
		\includegraphics[width=0.4\textwidth]{../../Bilder/PhilipsBild1.jpg}
		\hspace {1cm}
		\includegraphics[width=0.4\textwidth]{../../Bilder/PhilipsBild2.jpg}
	\end{center}
	\caption{Philips Semiconductors Sophia Antipolis}
	\label{fig:Philips Semiconductors Sophia Antipolis}
\end{figure}
\begin{itemize}
	\item 1987: VLSI Technology, ein kalifornisches Unternehmen, errichtet eine Niederlassung in Sophia  Antipolis. Am Anfang eher auf die Entwicklung von CAD Tools ausgelegt folgte schnell eine Gruppe die sich mit der Entwicklung von Wireless Business Applications besch�ftigt.
	\item 1995: Das Sophia Antipolis VLSI Zentrum vereint Entwicklung, Vermarktung, Logistik und Marketing unter einem Dach.
	\item 1999: VLSI Technology wird von Philips aufgekauft und geh�rt seitdem zu Philips Semiconductors.
	\item 2000: Philips Semiconductors wird Teil von Philips France.
\end{itemize}


Philips Semiconductors unterh�lt weltweit 18 Herstellungszentren, 30 Design Center und 100 Verkaufsb�ros. 
Philips Sophia Antipolis befindet sich im Technologiepark Sophia Antipolis im S�den Frankreichs.
Das Hauptaufgabengebiet der Firma liegt in der Entwicklung und Vermarktung von IC's f�r mobile Anwendungen wie GSM, GPRS, UMTS und Bluetooth.
Mit cirka 300 Mitarbeitern aus ungef�hr 15 L�ndern ist Philips Semiconductors in Sophia Antipolis ein internationales, multikulturelles Unternehmen. Ungef�hr 70 \% der Angestellten sind mit Forschung und Entwicklung besch�ftigt; 30 \% arbeiten in der Qualit�tssicherung, Logistik und im Marketing.\\
IC's (z.B.~EPROM,~Flash) werden von Philips Semiconductors in fortschrittlicher Low Power, Low Voltage CMOS - Technologie f�r Anwendungen in der (mobilen) Telekommunikation hergestellt. 
%\Tilde verhinder Trennung �(klebt W�rter zusammen)
Die Technologien zur Herstellung dieser hochkomplexen IC's werden dabei von Philips Semiconductors selbst entwickelt. Die zur Zeit verwendeten Technologien sind 0.25 $\mu$\textit{m}, 0.12 $\mu$\textit{m} und 0.18$\mu$\textit{m}. Die modernste 90 \textit{nm} Technologie wird seit Dezember 2003 f�r kleinere St�ckzahlen verwendet.